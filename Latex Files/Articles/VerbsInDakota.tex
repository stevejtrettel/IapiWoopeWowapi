% article example for classicthesis.sty
\documentclass[10pt,letter]{article} % KOMA-Script article scrartcl
\usepackage{lipsum}
\usepackage{fontspec}
\usepackage{url}
\usepackage[nochapters]{../classicthesis} % nochapters
\setmainfont{Gentium}
\newfontfamily\dak[]{Dak}





\begin{document}
    \title{\rmfamily\normalfont\spacedallcaps{Verbs In Dakota}}
    \date{} % no date
    \maketitle
    

    
    \section{The Role of Verbs in Dakota}

Verbs are the most important class of words in the Dakota language, and consequently the most complex.  Unlike in English, where verbs merely state the "action" which occurs in a sentence, Dakota verbs are full sentences, describing not only the action but also those who take part in it. In contrast to English where the verb specifies the action which occurs in a sentence, Dakota verbs specify the entire "core" of a sentence.  For example, the following verbs are actually one-word-sentences in Dakota. 

\begin{center}
\begin{tabular}{c c }
 \dak{wayáte} & You ate. \\
  \dak{wac̣íyanke}& I see you. \\
  \dak{wic̣áuƞk'upi}& We gave it to them.\\
\end{tabular}
\end{center}

By modifying verbs with appropriate affixes, they can take on even more specialized meanings, creating one-word sentences for more complex ideas.

\begin{center}
\begin{tabular}{c c }
  \dak{c̣icícaġe }& I made it for you.  \\
  \dak{wic̣úƞkiyawa} & We read it to them. \\
  \dak{nihdúżaża} & You washed yourself. \\
\end{tabular}
\end{center}

As the most important word in a sentence, Dakota has many different types of verbs (one for each type of sentence, really).  Even words which are usually thought of as adjectives in English are really verbs in Dakota, allowing simple one-word sentences to express states of being, as below.

\begin{center}
\begin{tabular}{c c }
 \dak{maháƞske} & I am tall.  \\
  \dak{niṭéca} & You are young. \\
 \dak{iṡtáuƞḣbapi} & We are sleepy. \\
\end{tabular}
\end{center}

While verbs in Dakota function very differently from verbs in English, nearly all of the differences stem from the fact that in Dakota verbs are stand-alone sentences and in English they are not.  Take for example the english sentence

\begin{center}
\emph{I saw you at the store this morning around 9 o'clock.}
\end{center}

One way to view this sentence is that there is a smaller "core" sentence living inside of it, namely "I saw you," and the rest of the sentence "at the store this morning around 9 o'clock" is just additional information tacked onto this core.  In Dakota, this would be expressed as

\begin{center}
\emph{Hahanna kin de napciwanka ape wahehanyan mazopiye ekta waciyanke.}
\end{center}

Where likewise, the sentence consists of a core sentence (the verb waciyanke, I saw you), and a bunch of descriptors adding additional information.  In this example, English and Dakota share the same core sentence (although as always, it is a single word in Dakota); but this is not always the case.  Consider the English sentence

\begin{center}
\emph{I read that book to him yesterday.}
\end{center}

Here the core sentence is "I read that book to him," and tacked onto it is the additional information of when; "yesterday."  In Dakota, this would be expressed

\begin{center}
\emph{Htanihan wowapi kin he wakiyawa.}
\end{center}

Here, as before the core of the sentence is the verb, wakiyawa, "I read it to him."  The additional information is when that occured, and what was read.  So, although we usually think of this sentence as having the meaning "I read that book to him yesterday," another way to translate it would be

\begin{center}
\emph{Yesterday, that book, I read it to him.}
\end{center}

This is the important difference - as verbs are whole sentences in Dakota \emph{everything} else is extra information added to the sentence. In Dakota, should we remove "wowapi kin he" from the above sentence, we are left with

\begin{center}
\emph{Htanihan wakiyawa.}
\end{center}

or "I read it to him yesterday," and the sentence is still grammatical (we removed an extra piece of information, not part of the core), whereas in English, removing "that book" from the sentence leaves "I read to him yesterday," which is still a grammatical sentence, but has a slightly different meaning.  Adding or removing extra descriptions from a sentence does not change the meaning, but adding or removing parts of the sentence's core does.  Dakota is actually much simpler than English in this sense, as every sentence's core is bound up in a one-word verb.

Because of this nice division of sentences into cores and extras, we will first concentrate on the cores of sentences, and later learn how to string on descriptions and extra information to flesh them out.


    \section{Types of Sentence Cores}

In English, the same verb can be used in multiple different ways, because the verb itself is only a portion of the core meaning of a sentence.  Consider for example the following two sentences;

\begin{center}
\emph{I ate this morning.}\\
\emph{I ate bread this morning.}
\end{center}

The core of the first sentence is "I ate," and the core of the second is "I ate bread."  To better see the difference in meaning between these two cores, consider these more verbose ways of saying the same things

\begin{center}
\emph{I was engaged in the process of consuming food.}\\
\emph{I took bread, put it in my mouth, chewed and swallowed.}
\end{center}

Since in Dakota all sentence cores are single words, there needs to be two different words to express these two different related meanings of "eat," and indeed there are!

\begin{center}
\begin{tabular}{ l r }
  wotA & to be engaged in the process of consuming food, to eat\\
  yutA & to chew and swallow something, to eat something \\
\end{tabular}
\end{center}

This kind of distinction is pervasive in Dakota, for every possible type of sentence core, there is a different verb to use.  (For those interested, the difference above is usually referred to as the difference between intransitive and transitive verbs).  Thus, oftentimes corresponding to a single english verb like "to eat," there will be multiple Dakota verbs, one for each more specialized meaning.

\begin{center}
\begin{tabular}{ l r }
  spanyA & to be cooking something\\
  waspanyA & to be in the process of cooking food \\
  wanyanka& to see something\\
  wawanyanka & to see things, to look around \\
\end{tabular}
\end{center}

With one verb per type of sentence core, we can break down Dakota verbs into different groups, based on what kind of sentence they correspond to.  One way to classify the difference between "I ate" and "I ate it" is by the number of things in the sentence core.  The first has an action, eating, and a single object, the person doing the eating.  The second has an action (again eating), but two objects; the person doing the eating and the thing being eaten.  These objects which are part of the sentence core are called the "arguments" of a verb, and different types of sentence cores have different numbers of arguments.



    \section{Types of Dakota Verbs}

Dakota verbs can have between 0 and 3 arguments, all of which are built into the verbs meaning.  For example "to be tall" is a verb of 1 argument, "to see something" is a verb of two arguments, and "to give something to somebody" is a verb of 3 arguments.


	\subsection{Zero  Argument Verbs}
In Dakota, there are sentences whose core is a pure action, with no participants involved.  These verbs don't need any arguments, as there's no one doing anything.  This is in contrast to English, where every sentence has to have a subject; these are truly "subjectless" verbs.  Good examples are words describing the weather;

\begin{center}
\begin{tabular}{ l r }
  magazu & it's raining\\
  maste & it's nice out \\
\end{tabular}
\end{center}

As these verbs describe something that is happening (but nobody is "doing"), English inserts a dummy subject, "it," to conform to the rules of grammar and make a complete sentence.  In Dakota there is no need for this; these verbs conjugate or need any modifications.





	\subsection{1 Argument Verbs}
Verbs of one argument are sentence cores that consist of an action and one participant. Examples are

\begin{center}
\begin{tabular}{ l r }
  hanske & he is tall \\
waspanwaye & I am cooking
\end{tabular}
\end{center}

To use these verbs as a full sentence, that one participant must be marked on the verb.  Thus, all verbs of one argument need one conjugation added onto them.  




	\subsection{2 Argument Verbs}
Verbs of two arguments are sentence cores that consist of an action and two participants.  Usually these two participants are the "do-er" of the action, and who/what the action is done too. To use these verbs as a full sentence then, they need to be conjugated twice, once for each person/object involved.

\marginpar{The do-er is usually called the agent, and the one things are done to is the patient}

\begin{center}
\begin{tabular}{ l r }
  wamayadake & You see me \\
wowicawak'u & I fed them
\end{tabular}
\end{center}

    
	\subsection{3+ Argument Verbs}
Some sentence cores involve an action with more than two participants.  An example of such an action in English is "to give," which involves three seperate things: the giver, the receiver, and the object given.  Since in Dakota verbs have all their arguments built-in, the equivelant verb \emph{k'u} has three arguments, and as a stand-alone sentence means "he gave it to her"
\marginpar{k'u could also be translated as "she gave it to him," as in Dakota there are not special words to differentiate between the gender of a pronoun, like in English}

Taking "to give" as an example, almost every sentence that uses it involves a person giving a thing to a person.  That is, there are usually only two people involved.  (Of course, you could say something like "I gave you to them" which involves three groups of people, but it's not nearly as common).  Because almost always verbs of 3 or more arguments have only two human participants, Dakota only requires conjugation for two arguments on the verb.  The remaining argument is understood to be "it" in most cases, and can be specified in the sentence if need be.

\begin{center}
\begin{tabular}{ l r }
 mayak'u &you gave it to me \\
wowapi kin de mayak'u & you gave this book to me
\end{tabular}
\end{center}

Some other examples of 3-argument verbs are

\begin{center}
\begin{tabular}{ l r }
 wakiyawa & I read it to her \\
spanwicayeciciye & you cooked it for them\\
wicaunkipazopi & we showed it to them
\end{tabular}
\end{center}


\section{Summary}
Verbs in Dakota are sentences by themselves, and adding any additional words provides details or clairification, but isn't necessary for the sentence to be grammatically correct.  Viewing each verb as a one-word sentence, it already includes information on all of the participants involved in that action, which can be 0 for weather, or actions which happen but nobody causes, 1 for states of being or activities done, 2 for actions done to something, or 3 for actions done to something for somebody.  Even though more than two participants may be needed for a particular verb, only two of them (the two animate ones usually) are marked with different conjugations.

    % bib stuff
    \nocite{*}
    \addtocontents{toc}{\protect\vspace{\beforebibskip}}
    \addcontentsline{toc}{section}{\refname}    
    \bibliographystyle{plain}
    \bibliography{../Bibliography}
\end{document}