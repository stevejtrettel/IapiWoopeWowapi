% article example for classicthesis.sty
\documentclass[10pt,a4paper]{article} % KOMA-Script article scrartcl
\usepackage{lipsum}
\usepackage{url}
\usepackage[nochapters]{../classicthesis} % nochapters

\begin{document}
    \title{\rmfamily\normalfont\spacedallcaps{Conjugating Verbs}}
    \date{} % no date
    \maketitle
    

Depending on how many arguments a verb has, it conjugates 0, 1 or 2 times.  These conjugatons take the place of pronouns in English, which is what allows Dakota sentence cores to be single words.  For example the verb expressing "walking" in Dakota has a single argument (the person/people doing the walking).  Dakota has available seven different conjugations, which when added to mani can change its meaning.
\marginpar{these conjugations are usually called "personal affixes," meaning they mark the person(s) doing the action, and are fixed onto the verb}

\begin{center}
\begin{tabular}{ l r }
  mani & He is walking. \\
  manipi & They are walking.\\
  mawani & I am walking. \\
  mayani & You are walking. \\
  mayanipi & You guys are walking.\\
  maunni & You and I are walking. \\
  maunnipi & We are walking. \\
\end{tabular}
\end{center}

Later chapters will cover how to conjugate each specific type of verb correctly, so here we will really just talk more about the big picture, what each type of conjugation specifies, where they are placed in verbs, and which verbs get which kind of conjugations.



     \section{Types of Conjugations}

As we saw above, Dakota has 7 different personal affixes that can be attached, which break up into 4 main groups. These groups are the "I" group, the "We" group, the "You" group, and the "They" group.  

\marginpar{The usual terms for groups of pronouns are first, second and third person.  First Person = I, we.  Second Person= You, Third Person=he, she, it, they.}

Each of these groups has two conjugations in it, a singular and a plural (except the "I" group of course!), which makes a total of 7.  To make things even easier, the plural forms are made by tacking the ending \emph{pi} right onto the end of the singular form, so there's really only 4 different things to remember!


\subsection{The "They" Conjugations}
The easiest case is verbs which you want to conjugate for he, she, it or they.  The basic form of any Dakota verb is a full sentence by itself, with the meaning that the subject / object of the sentence is third-person singular.  Thus, for the singular version of this group, you don't need to do anything!  Even though when learning a new verb (say, wanyanke) we often think of it as just the action abstractly, it really has the meaning of "he, she, or it" built right in.
\begin{center}
hanske\\
he is tall, 
she is tall,
it is tall
\end{center}

To get the plural form, all we need to do is add \emph{pi} to the verb above.  However, remember that \emph{pi} stops the last vowel of ablaut words from turning into an \emph{e}, so the words revert back to their form ending in \emph{a}.

\begin{center}
hanskapi\\
they are tall
\end{center}




\subsection{The "I" conjugations}
Marking a verb with an "I" conjugation does exactly what you think it would--it adds the meaning of "I" or "me" to the verb.  Depending on the verb, there are basically three different conjugations that could be used.

\begin{center}
\begin{tabular}{l c r }
 &"I" Conjugations & \\
 wa & ma & bd \\
\end{tabular}
\end{center}

Examples of each of these can be found with the verbs \emph{mani}, to walk, \emph{hanske}, to be tall, and \emph{wayatke}, to drink.

\begin{center}
\begin{tabular}{c c c }
 mawani & mahanske & wabdatke \\
I walk & I am tall & I am drinking \\
\end{tabular}
\end{center}



\subsection{The "We" Conjugations}
There are two forms of "we" in Dakota, one that specifically means "you and I," and the other meaning \emph{we} more generally, as in the speaker and some other people, no matter who they are or how many.  The basic conjugation for the "We" group is \emph{un}, which gives us the following forms of \emph{hanske}

\begin{center}
\begin{tabular}{c c  }
unhanske & unhanskapi\\
You and I are tall & We are tall\\
\end{tabular}
\end{center}

\subsection{The "You" Conjugations}
There are two different conjugations in the "you" group in Dakota - a singular you and a plural you.  While English doesn't technically have a singular and plural version, it's easiest to think of "you" being the singular and "you guys" being the plural.  Like for "I" there are three basic types of conjugations here








     \section{Place of Conjugation}

\subsection{Verbs with Initial Stress}

\subsection{Causative Verbs}


    \section{Active / Stative Distinction}


     \section{Summary}





    % bib stuff
    \nocite{*}
    \addtocontents{toc}{\protect\vspace{\beforebibskip}}
    \addcontentsline{toc}{section}{\refname}    
    \bibliographystyle{plain}
    \bibliography{../Bibliography}
\end{document}